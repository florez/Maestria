\documentclass{article}

\usepackage[utf8]{inputenc}
\usepackage[spanish]{babel}
\usepackage{physics}
\usepackage{amssymb}
\usepackage{hyperref}
\usepackage{tikz}

\title{Mecánica Estadística\\Tarea 5: Teoría de Landau y Renormalización}
\author{Iván Mauricio Burbano Aldana}

\begin{document}

\maketitle

\section{Teoría de Landau}

\subsection{Parámetro de orden}

\begin{enumerate}

\item La derivada funcional de $S$ es
\begin{align}
\begin{split}
\fdv{S}{\phi(\vb{r})}[\phi]=&\fdv{H}{\phi(\vb{r})}[\phi]-\fdv{\phi(\vb{r})}\int \dd[d]{\vb{r'}} h(\vb{r'})\phi(\vb{r'})\\
=&\fdv{\phi(\vb{r'})}\int\dd[d]{\vb{r'}}\qty(\frac{1}{2}\|\grad{\phi}(\vb{r'})\|^2+A\phi(\vb{r'})^2+B\phi(\vb{r'})^4)\\
&-\int\dd[d]{\vb{r'}}h(\vb{r'})\delta(\vb{r'}-\vb{r})\\
=&\int\dd[d]{\vb{r'}}\grad{\phi}(\vb{r'})\vdot\grad{\delta}(\vb{r'}-\vb{r})\\
&+\int\dd[d]{\vb{r'}}\qty(2A\phi(\vb{r'})\delta(\vb{r'}-\vb{r})+4B\phi(\vb{r'})^3\delta(\vb{r'}-\vb{r}))-h(\vb{r})\\
=&\int\dd[d]{\vb{r'}}\grad{\phi}(\vb{r'})\vdot\grad{\delta}(\vb{r'}-\vb{r})+2A\phi(\vb{r})+4B\phi(\vb{r})^3-h(\vb{r}).
\end{split}
\end{align}
Note que según la definición de derivada de una distribución se tiene
\begin{align}
\begin{split}
\grad{\phi}(\vb{r'})\vdot\grad{\delta}(\vb{r'}-\vb{r})=&\sum_{i=1}^d\partial_i\phi(\vb{r'})\partial_i{\delta}(\vb{r'}-\vb{r})=-\sum_{i=1}^d\partial_i^2\phi(\vb{r'})\delta(\vb{r'}-\vb{r})\\
=&-\Delta\phi(\vb{r'})\delta(\vb{r'}-\vb{r}).
\end{split}
\end{align}
Se concluye entonces que
\begin{align}
\begin{split}
\fdv{S}{\phi(\vb{r})}[\phi]=&\int\dd[d]{\vb{r'}}\grad{\phi}(\vb{r'})\vdot\grad{\delta}(\vb{r'}-\vb{r})+2A\phi(\vb{r})+4B\phi(\vb{r})^3-h(\vb{r})\\
=&-\int\dd[d]{\vb{r'}}\Delta\phi(\vb{r'})\delta(\vb{r'}-\vb{r})+2A\phi(\vb{r})+4B\phi(\vb{r})^3-h(\vb{r})\\
=&-\Delta\phi(\vb{r})+2A\phi(\vb{r})+4B\phi(\vb{r})^3-h(\vb{r}).
\end{split}
\end{align}
En efecto, la condición de minimización $0=\fdv{S}{\phi(\vb{r})}[\phi_0]$ toma la forma
\begin{equation}
h(\vb{r})=2A\phi_0(\vb{r})+4B\phi_0(\vb{r})^3-\Delta\phi_0(\vb{r}).
\end{equation}

\item En el caso $h=0$ en el que $\phi_0$ es constante, $\Delta\phi=0$ y por lo tanto
\begin{equation}
0=2A\phi_0+4B\phi_0^3=2\phi_0(A+2B\phi_0^2).
\end{equation}
Se obtienen las soluciones $\phi_0=0$ y, si $A$ y $B$ tienen signos opuestos, $\phi_0=\pm\sqrt{-\frac{A}{2B}}$.

\item Empezamos considerando las expansiones en series de potencias de $A$ y $B$ alrededor de la temperatura crítica $T_c$
\begin{align}
\begin{split}
A(T)=&A_0+A_1(T-T_c)+\cdots\\
B(T)=&B_0+B_1(T-T_c)+\cdots.
\end{split}
\end{align}
Recordando los resultado de campo medio, queremos dependencia del tiempo. Por lo tanto no podemos permitirnos considerar a $A$ y $B$ simultaneamente como constantes. Mas aún, ya que en la aproximación de campo medio tenemos para $T<T_c$ la relación $m\propto|T-T_c|^{1/2}$ y, como se demostró en clase, podemos identificar $\phi_0$ con $m$, observamos que las soluciones
\begin{equation}
\phi_0=\pm\sqrt{-\frac{A_0+A_1(T-T_c)+\cdots}{2B_0+B_1(T-T_c)+\cdots}}
\end{equation}
coinciden con las obtenidas en campo medio al poner $A_0=A_2=\cdots=B_1=B_2\cdots=0$. Más aún, queremos que esta solución sea inválida en el caso $T>T_c$. Esto lo podemos obtener asegurando que los coeficientes $A_1$ y $B_0$ tengan el mismo signo. Las escogemos positivas. En este caso, si $T>T_c$, se tiene $A>0$. Por lo tanto, la única solución posible es $\phi_0=0$. Por el otro lado, si $T<T_c$, tanto la solución $\phi_0=0$ como las soluciones 
\begin{equation}\label{ec:magnetizacion_ferromagnetica}
\phi_0=\pm\sqrt{\frac{A_1(T_c-T)}{B}}
\end{equation}
son permitidas. Sin embargo, solo consideramos la solución \eqref{ec:magnetizacion_ferromagnetica} ya que esta minimiza la energía libre. En efecto, ya que $h=0$, se tiene
\begin{align}
\begin{split}
F=&H[\phi_0]=\int\dd[d]{\vb{r}}\qty(A\phi^2+B\phi^4)=\int\dd[d]{\vb{r}}\qty(A+B\phi^2)\phi^2\\
=&\begin{cases}
0, & \phi_0=0\\
-\int\dd[d]{\vb{r}}(A-B\frac{A}{2B})\frac{A}{2B}=-\int\dd[d]{\vb{r}}\frac{A^2}{4B}<0, & \phi_0=\pm{\sqrt{-\frac{A}{2B}}}.
\end{cases}
\end{split}
\end{align} 

\end{enumerate}

\subsection{Correlaciones}

\begin{enumerate}

\item Tomando la derivada funcional y utilizando la relación vista en clase $G(\vb{r'},\vb{r})=\fdv{h(\vb{r})}\phi_0(\vb{r'})$ se tiene
\begin{align}
\begin{split}
\delta(\vb{r'}-\vb{r})=&\fdv{h(\vb{r})}h(\vb{r'})=\fdv{h(\vb{r})}\qty(2A\phi_0(\vb{r'})+4B\phi_0(\vb{r'})^3-\Delta\phi_0(\vb{r'}))\\
=&2A\fdv{h(\vb{r})}\phi_0(\vb{r'})+12B\phi_0(\vb{r'})^2\fdv{h(\vb{r})}\phi_0(\vb{r'})-\Delta\fdv{h(\vb{r})}\phi_0(\vb{r'})\\
=&2AG(\vb{r'},\vb{r})+12B\phi_0(\vb{r'})^2G(\vb{r'},\vb{r})-\Delta_{\vb{r'}} G(\vb{r'},\vb{r}).
\end{split}
\end{align}
Si $h=0$ y el sistema es homogeneo, $\phi_0$ es constante y $G(\vb{r'},\vb{r})=G(\vb{r'}-\vb{r})$. Es claro entonces por la regla de la cadena que
\begin{equation}
\Delta_{\vb{r'}} G(\vb{r'},\vb{r})=\Delta_{\vb{r'}} G(\vb{r'}-\vb{r})=\Delta G(\vb{r'}-\vb{r}).
\end{equation} 
Tomando
\begin{equation}
\xi^{-2}=2A+12B\phi_0^2
\end{equation}
y haciendo el cambio de variable $\vb{r'}-\vb{r}\mapsto\vb{r}$ se concluye
\begin{equation}\label{ec:funcion_green}
\delta(\vb{r})=(-\Delta+\xi^{-2})G(\vb{r}).
\end{equation}
Podemos determinar
\begin{align}
\begin{split}
\xi =& \frac{1}{\sqrt{2A+12B\phi^2}}\\
=&\begin{cases}
\frac{1}{\sqrt{2A-12B\frac{A}{2B}}}=\frac{1}{2\sqrt{-A}}=\frac{1}{2\sqrt{A_1(T_c-T)}}, & T<T_c\\
\frac{1}{\sqrt{2A}}=\frac{1}{\sqrt{2A_1(T-T_c)}}, & T>T_c.
\end{cases}
\end{split}
\end{align}

\item Utilizando la forma integral de la distribución delta e insertando la transformada de Fourier en \eqref{ec:funcion_green} se tiene
\begin{align}
\begin{split}
\int\frac{\dd[d]{\vb{k}}}{(2\pi)^d}e^{i\vb{k}\vdot\vb{r}}=&\delta(\vb{r})=(-\Delta+\xi^{-2})\int\frac{\dd[d]{\vb{k}}}{(2\pi)^d}e^{i\vb{k}\vdot\vb{r}}\tilde{G}(\vb{k})\\
=&\int\frac{\dd[d]{\vb{k}}}{(2\pi)^d}(-\Delta+\xi^{-2})e^{i\vb{k}\vdot\vb{r}}\tilde{G}(\vb{k})\\
=&\int\frac{\dd[d]{\vb{k}}}{(2\pi)^d}(\vb{k}^2+\xi^{-2})e^{i\vb{k}\vdot\vb{r}}\tilde{G}(\vb{k})
\end{split}
\end{align}
en vista de que $\Delta e^{i\vb{k}\vdot\vb{r}}=i\vb{k}\vdot i\vb{k}e^{i\vb{k}\vdot\vb{r}}=-\vb{k}^2e^{i\vb{k}\vdot\vb{r}}$. Comparando se obtiene que 
\begin{equation}
\tilde{G}(\vb{k})=\frac{1}{\vb{k}^2+\xi^{-2}}.
\end{equation} 

\item Utilizando el resultado anterior y restringiendo a $d=3$ tenemos 
\begin{align}
\begin{split}
G(\vb{r})=&\int\frac{\dd[3]{\vb{k}}}{(2\pi)^3}\frac{e^{i\vb{k}\vdot\vb{r}}}{\vb{k}^2+\xi^{-2}}\\
=&\frac{1}{(2\pi)^3}\int_0^\infty\dd{u}\int_0^\pi\dd{\theta}\int_0^{2\pi}\dd{\phi}u^2\sin(\theta)\frac{e^{iu\|\vb{r}\|\cos(\theta)}}{u^2+\xi^{-2}}\\
=&\frac{1}{(2\pi)^2}\int_0^\infty\dd{u}\int_1^{-1}\dd{v}(-1)u^2\frac{e^{iu\|\vb{r}\|v}}{u^2+\xi^{-2}}\\
=&\frac{1}{i\|\vb{r}\|(2\pi)^2}\int_0^\infty\dd{u}u\frac{e^{iu\|\vb{r}\|}-e^{-iu\|\vb{r}\|}}{u^2+\xi^{-2}}
\end{split}
\end{align}
Note que bajo un cambio de variable
\begin{align}
\begin{split}
\int_0^\infty\dd{u}u\frac{e^{-iu\|\vb{r}\|}}{u^2+\xi^{-2}}=&\int_0^{-\infty}\dd{u}(-1)(-u)\frac{e^{iu\|\vb{r}\|}}{(-u)^2+\xi^{-2}}\\
=&-\int_{-\infty}^0\dd{u}u\frac{e^{iu\|\vb{r}\|}}{u^2+\xi^{-2}}.
\end{split}
\end{align}
Por lo tanto
\begin{equation}
G(\vb{r})=\frac{1}{i\|\vb{r}\|(2\pi)^2}\int_{-\infty}^\infty\dd{u}\frac{ue^{iu\|\vb{r}\|}}{u^2+\xi^{-2}}.
\end{equation}
Sea $R>i\xi^{-1}$. Considere bajo la orientación en contra de las manecillas del reloj los conjuntos 
\begin{align}
\begin{split}
S^+_R:=&\{Re^{i\theta}\in\mathbb{C}|\theta\in[0,\pi]\}\subseteq D_R:=S^+_R\cup[-R,R]\times\{0\}\\
\subseteq&\mathbb{C}\setminus\{i\xi^{-1},-i\xi^{-1}\},
\end{split}
\end{align}
el dominio de definición de la función holomorfa
\begin{align}
\begin{split}
f:\mathbb{C}\setminus\{i\xi^{-1},-i\xi^{-1}\}&\rightarrow\mathbb{C}\\
z&\mapsto \frac{ze^{iz\|\vb{r}\|}}{z^2+\xi^{-2}}=\frac{ze^{iz\|\vb{r}\|}}{(z+i\xi^{-1})(z-i\xi^{-1})}.
\end{split}
\end{align}
$D_R$ solo encierra el polo $i\xi^{-1}$ y en este punto el residuo de $f$ claramente es
\begin{equation}
\frac{i\xi^{-1}e^{ii\xi^{-1}\|\vb{r}\|}}{2i\xi^{-1}}=\frac{1}{2}e^{-\|\vb{r}\|/\xi}.
\end{equation}
Por lo tanto por el teorema del residuo
\begin{equation}
\int_{D_R}\dd{z}f(z)=2\pi i\frac{1}{2}e^{-\|\vb{r}\|/\xi}=i\pi e^{-\|\vb{r}\|/\xi},
\end{equation}
y
\begin{align}
\begin{split}
\int_{-\infty}^\infty \dd{u}f(u)=&\lim_{R\rightarrow\infty}\qty(\int_{D_R}\dd{z}f(z)-\int_{S^+_R}\dd{z}f(z))\\
=&i\pi e^{-\|\vb{r}\|/\xi}-\lim_{R\rightarrow\infty}\int_{S^+_R}\dd{z}f(z)=i\pi e^{-\|\vb{r}\|/\xi}.
\end{split}
\end{align}
El límite desaparece por la estimación
\begin{align}
\begin{split}
\left\vert\int_{S^+_R}\dd{z}f(z)\right\vert\leq&\int_{S^+_R}\dd{z}|f(z)|=\int\dd{z}R\left\vert \frac{e^{iRre^{i\theta}}}{R^2e^{i2\theta}+\xi^{-2}}\right\vert\\
=&R\int\dd{z}\frac{e^{-Rr\sin(\theta)}}{(R^2e^{i2\theta}+\xi^{-2})(R^2e^{-i2\theta}+\xi^{-2})}\\
=&R\int\dd{z}\frac{e^{-Rr\sin(\theta)}}{R^4+\xi^{-4}+2R^2\xi^{-2}\cos(2\theta)}\\
\leq&\pi R^2\sup\qty{\left.\frac{e^{-Rr\sin(\theta)}}{R^4+\xi^{-4}+2R^2\xi^{-2}\cos(2\theta)}\right\vert\theta\in\qty(0,\pi)}\\
\underset{R\rightarrow\infty}{\longrightarrow}&\pi R^2 0=0.
\end{split}
\end{align}
Para poner los límites del ángulo utilizamos el hecho de que $[0,\pi]\setminus(0,\pi)=\{0,\pi\}$ tiene medida nula. Se concluye que
\begin{equation}\label{ec:correlacion_3}
G(\vb{r})=\frac{1}{i\|\vb{r}\|(2\pi)^2}i\pi e^{-\|\vb{r}\|/\xi}=\frac{e^{-\|\vb{r}\|/\xi}}{4\pi\|\vb{r}\|}
\end{equation}

\item En el caso $d=2$ tenemos tomando coordenadas polares con los ángulos medidos desde $\vb{r}$
\begin{align}
\begin{split}
G(\vb{r})=&\int\frac{\dd[2]{\vb{k}}}{(2\pi)^2}\frac{e^{i\vb{k}\vdot\vb{r}}}{\vb{k}^2+\xi^{-2}}\\
=&\frac{1}{(2\pi)^2}\int_0^\infty\dd{u}\int_0^{2\pi}\dd{\theta}u\frac{e^{iu\|\vb{r}\|\cos(\theta)}}{u^2+\xi^{-2}}.
\end{split}
\end{align}
Recordando la expresión integral para la función de Bessel de orden $0$\cite{NIST_DLMF_10.32}
\begin{align}
\begin{split}
J_0(x)=&\frac{1}{2}(J_0(x)+J_0(x))\\
=&\frac{1}{2\pi}\qty(\int_0^\pi \dd{\theta}e^{ix\cos(\theta)}+\int_0^\pi \dd{\theta}e^{-ix\cos(\theta)})\\
=&\frac{1}{2\pi}\qty(\int_0^\pi \dd{\theta}e^{ix\cos(\theta)}+\int_\pi^2\pi \dd{\theta}e^{-ix\cos(\theta-\pi)})\\
=&\frac{1}{2\pi}\qty(\int_0^\pi \dd{\theta}e^{ix\cos(\theta)}+\int_\pi^2\pi \dd{\theta}e^{ix\cos(\theta)})\\
=&\frac{1}{2\pi}\int_0^{2\pi} \dd{\theta}e^{ix\cos(\theta)}
\end{split}
\end{align}
se tiene
\begin{equation}
G(\vb{r})=\frac{1}{2\pi}\int_0^\infty\dd{u}\frac{uJ_0(u\|\vb{r}\|)}{u^2+\xi^{-2}}.
\end{equation}
Recordando que bajo ciertas asunciones sobre el comportamiento de las funciones se tiene\cite{Williams1973}
\begin{equation}
\int_0^\infty \dd{x}f(x)g(x)=\int_0^\infty \mathcal{L}(f)(x)\mathcal{L}^{-1}(g)(x),
\end{equation}
se tiene que\cite{NIST_DLMF_10.32, Williams1973}
\begin{align}
\begin{split}
G(\vb{r})=&\frac{1}{2\pi}\int_0^\infty\dd{u}\mathcal{L}(u\mapsto J_0(u\|\vb{r}\|))(u)\mathcal{L}^{-1}\qty(u\mapsto\frac{u}{u^2+\xi^{-2}})(u)\\
=&\frac{1}{2\pi}\int_0^\infty\dd{u}\frac{1}{\sqrt{\|\vb{r}\|^2+u^2}}\cos(\frac{u}{\xi})\\
=&\frac{1}{2\pi}\int_0^\infty\dd{u}\|\vb{r}\|\frac{1}{\sqrt{\|\vb{r}\|^2+\|\vb{r}\|^2u^2}}\cos(\frac{\|\vb{r}\|u}{\xi})\\
=&\frac{1}{2\pi}\int_0^\infty\dd{u}\frac{\cos(\frac{\|\vb{r}\|u}{\xi})}{\sqrt{1+u^2}}=K_0(\|\vb{r}\|/\xi)
\end{split}
\end{align}
El crédito de esta solución va para Iwaniuk\cite{Iwaniuk2018}.

\item En el caso $d=2$ tenemos según el apéndice del enunciado que para $\|\vb{r}\|/\xi\gg 1$, es decir, $\|\vb{r}\|\gg \xi$ se tiene
\begin{equation}
G(\vb{r})=\frac{1}{2\pi}K_0(\|\vb{r}\|/\xi)\sim\frac{1}{2\pi}\sqrt{\frac{\pi\xi}{2\|\vb{r}\|}}e^{-\|\vb{r}\|/\xi}.
\end{equation}
Este es el comportamiento deseado. En el caso $d=3$ el comportamiento es trivial de \eqref{ec:correlacion_3}.

\item En el caso $d=2$ cuando $\xi\gg\|\vb{r}\|$ se tiene $\|\vb{r}\|/\xi\ll 1$. Por lo tanto, según el apéndice del enunciado se tiene
\begin{equation}
G(\vb{r})\sim-\frac{1}{2\pi}\ln(\|\vb{r}\|/\xi).
\end{equation}
Ya que no hay dependencia en potencias, se tiene $0=d+\eta-2=2+\eta-2=\eta$. En el caso $d=3$ vemos que 
\begin{equation}
G(\vb{r})=\frac{e^{-\|\vb{r}\|/\xi}}{4\pi\|\vb{r}\|}\sim\frac{1}{2\pi\|\vb{r}\|}.
\end{equation}
Por lo tanto $1=d+\eta-2=3+\eta-2=1+\eta$. Se concluye una vez más que $\eta=0$.

\item En el punto crítico $\xi\rightarrow\infty$. Por lo tanto en caso $d=2$ y $d=3$
\begin{equation}
\tilde{G}(\vb{r})=\frac{1}{\vb{k}^2+\zeta}\sim\frac{1}{\vb{k}^2}=\|\vb{k}\|^{-2}
\end{equation}
Se concluye entonces que $\eta=0$ en ambos casos, resultado que concuerda con el anterior.

\end{enumerate}

\section{Renormalización}

\begin{figure}[h]
\begin{center}
\begin{tikzpicture}
\foreach \x in {0,...,5}
\foreach \y in {0,...,2}
{
\draw [fill] (2*\x,{2*sqrt(3)*\y}) circle [radius=0.08];
\draw [fill] (2*\x+1,{2*sqrt(3)*\y+sqrt(3)}) circle [radius=0.08];
}
\foreach \x in {0,3}
\foreach \y in {0,...,2}
{
\draw [dashed] (2*\x,{2*sqrt(3)*\y}) -- (2*\x+1,{2*sqrt(3)*\y+sqrt(3)}) -- (2*\x+2,{2*sqrt(3)*\y}) -- (2*\x,{2*sqrt(3)*\y});
\draw [dashed] (2*\x+3,{2*sqrt(3)*\y+sqrt(3)}) -- (2*\x+5,{2*sqrt(3)*\y+sqrt(3)}) -- (2*\x+4,{2*sqrt(3)*\y+2*sqrt(3)}) -- (2*\x+3,{2*sqrt(3)*\y+sqrt(3)});
}
\draw [fill] (4,{6*sqrt(3)}) circle [radius=0.08];
\draw [fill] (10,{6*sqrt(3)}) circle [radius=0.08];
\draw [ultra thick] (1,{sqrt(3)}) -- (0,{sqrt(3)*2});
\draw [ultra thick] (1,{sqrt(3)}) -- (2,{sqrt(3)*2});
\draw [ultra thick] (5,{3*sqrt(3)}) -- (6,{sqrt(3)*2});
\draw [ultra thick] (5,{3*sqrt(3)}) -- (7,{sqrt(3)*3});
\draw [ultra thick] (9,{5*sqrt(3)}) -- (8,{sqrt(3)*4});
\draw [ultra thick] (9,{5*sqrt(3)}) -- (7,{sqrt(3)*5});
\end{tikzpicture}
\end{center}
\caption{\label{fig:red}Se muestra una red triangular de sitios para un retículo de espines. Las lineas punteadas señalan la agrupación de los espines en el equema de renormalización propuesto. Las lineas solidas corresponden a ejemplos de vecinos más cercanos que pertencen a grupos distintos.}
\end{figure}

\begin{enumerate}

\item Considere la red triangular mostrada en la figura \ref{fig:red}. Al conjunto contable de sitios lo vamos a denotar $\Lambda$ y al de vecinos más cercanos $\ev{\Lambda}\subseteq\Lambda\times\Lambda$. Cabe notar que solo escojemos un representante por vecino, es decir, si $(i,j)\in\ev{\Lambda}$ entonces $(j,i)\not\in\ev{\Lambda}$. Entonces los microestados de la red se pueden tomar como los elementos $s\in\{-1,1\}^\Lambda$ bajo la interpretación de que $s_i:=s(i)$ para $i\in\Lambda$ es el espín en el sitio $i$. El Hamiltoniano toma la forma
\begin{align}
\begin{split}
\beta H:\{-1,1\}^\Lambda&\rightarrow \mathbb{R}\\
s&\mapsto \beta H(s)=-K\sum_{(i,j)\in\ev{\Lambda}}s_is_j-h\sum_{i\in\Lambda}s_i
\end{split}
\end{align} 
para unas constantes $K,h\in[0,\infty)$.

Vamos a considerar un esquema de renormalización agrupando los sitios en grupos de 3. Por lo tanto, el factor de escala $\lambda$ está determinado por la restricción de que $\lambda^2=3$. Se concluye que $\lambda=\sqrt{3}$. Podemos tomar $\Lambda'\subseteq \mathcal{P}(\Lambda)$ como el conjunto de los grupos de 3 sitios. Este va a ser el conjunto de sitios despues del procedimiento de renormalización. Note que para cada $\mu\in\Lambda'$ se tiene un conjunto $\ev{\mu}:=\{(i,j)\in\ev{\Lambda}|i,j\in\mu\}\subseteq\ev{\Lambda}$ de vecinos contenidos en $\mu$. Ya que los elementos de $\Lambda'$ son disjuntos,
\begin{equation}
\beta H(s)=\beta H_0(s)+\beta V(s)
\end{equation}
donde
\begin{equation}
\beta H_0(s)=-K\sum_{(i,j)\in\bigcup\limits_{\mu\in\Lambda'}\ev{\mu}}s_is_j=-K\sum_{\mu\in\Lambda'}\sum_{(i,j)\in\ev{\mu}}s_is_j
\end{equation}
y
\begin{equation}
\beta V(s)=-K\sum_{(i,j)\in\ev{\Lambda}\setminus\bigcup\limits_{\mu\in\Lambda'}\ev{\mu}}s_is_j+h\sum_{i\in\Lambda}s_i.
\end{equation}

Note que el conjunto $\ev{\Lambda}\setminus\bigcup\limits_{\mu\in\Lambda'}\ev{\mu}$ es el de vecinos más cercanos que corresponden a grupos distintos. Sea $\ev{\Lambda'}\subseteq\Lambda'\times\Lambda'$ el conjunto de grupos vecinos más cercanos. Sin perdida de generalidad, ordenamos los pares de manera que si $(\mu,\nu)\in\ev{\Lambda'}$ el segmento más corto entre los triángulos generados por $\mu$ y $\nu$ tiene un punto final en un vertice del generado por $\mu$ y otro en la mitad de una arista del generado por $\nu$. Defina $s_{(\mu,\nu)}$ como el espín en el vertice en el segmento y $s_{(\mu,\nu)}^{(1)}$ y $s_{(\mu,\nu)}^{(2)}$ como los espinos en los vertices que generan la arista en el segmento. En particular $s_{(\mu,\nu)}=s_i$ para algún $i\in\mu$ y $s_{(\mu,\nu)}^{(1)}=s_{j_1}$ y $s_{(\mu,\nu)}^{(2)}=s_{j_1}$ para algunos $j_1,j_2\in\nu$ distintos. La figura \ref{fig:red} muestra que todos los $s_is_j$ con $(i,j)\in\ev{\Lambda}\setminus\bigcup\limits_{\mu\in\Lambda'}\ev{\mu}$ son de la forma $s_{(\mu,\nu)}s_{(\mu,\nu)}^{(1)}$ o $s_{(\mu,\nu)}s_{(\mu,\nu)}^{(2)}$ para $\mu,\nu\in\ev{\Lambda'}$. Concluimos que 
\begin{equation}
\beta V(s)=-K\sum_{(\mu,\nu)\in\ev{\Lambda'}}\qty(s_{(\mu,\nu)}s_{(\mu,\nu)}^{(1)}+s_{(\mu,\nu)}s_{(\mu,\nu)}^{(2)})+h\sum_{i\in\Lambda}s_i.
\end{equation}

Defina la función 
\begin{align}
\begin{split}
f:\{-1,1\}^3&\rightarrow\{-1,1\}\\
(s_1,s_2,s_3)&\mapsto s_1+s_2+s_3.
\end{split}
\end{align}
Si $\mu=\{i,j,k\}$ entonces interpretamos $s'_\mu:=f(s_i,s_j,s_k)$ como el espín en el sitio $\mu$ posterior al proceso de renormalización. Es claro que este mapa le asigna a los nuevos sitios su espín según una regla de mayoría. Esta función se extiende a
\begin{align}
\begin{split}
g:\{-1,1\}^\Lambda&\rightarrow\{-1,1\}^{\Lambda'}\\
s&\mapsto s'
\end{split}
\end{align}
donde $s'(\mu):=s'_\mu$. Este mapa le asigna a cada estado de la red original un estado en la red renormalizada. Mediante este podemos inducir a partir de la medida de probabilidad canónica sobre $\{-1,1\}^\Lambda$ generada por
\begin{equation}
P(\{s\})=\frac{1}{Z}e^{-\beta H(s)},
\end{equation}
con $Z$ la constante de normalización apropiada, una medida de probabilidad sobre $\{-1,1\}^{\Lambda'}$ generada por
\begin{equation}
P'(\{s'\})=P(g^{-1}(\{s'\}))=\frac{1}{Z}\sum_{s\in g^{-1}(\{s'\})}e^{-\beta H(s)}
\end{equation}

\end{enumerate}

\bibliography{/Users/ivan/Documents/Bib_Files/library}
\bibliographystyle{ieeetr}

\end{document}
