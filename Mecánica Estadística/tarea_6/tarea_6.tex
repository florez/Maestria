\documentclass{article}

\usepackage[utf8]{inputenc}
\usepackage[spanish]{babel}
\usepackage{amssymb}
\usepackage{physics}

\title{Mecánica Estadística\\
Tarea 6: Mecánica estadística fuera del equilibrio}
\author{Iván Mauricio Burbano Aldana}

\begin{document}

\maketitle

\section{Relaciones de Kramers-Kronig}

\begin{enumerate}

\item Considere $z=x+iy\in\mathbb{R}\times\mathbb{R}^+\subseteq\mathbb{C}$ en el plano complejo superior. Entonces
\begin{equation}
\hat{\chi}(z)=\int_0^\infty\chi(t)e^{izt}\dd{t}=\int_0^\infty\chi(t)e^{-yt}e^{ixt}dt=\int_0^\infty\chi(t)e^{ixt}e^{-yt}\dd{t}
\end{equation} 
existe ya que para todo $y\in\mathbb{R}^+:=(0,\infty)$
\begin{equation}
\int_0 ^\infty |\chi(t)e^{-yt}| \dd{t}=\int_0 ^\infty |\chi(t)||e^{-yt}| dt\leq\int_0 ^\infty |\chi(t)| \dd{t}<\infty.
\end{equation}
En efecto, por hipótesis $\chi$ es medible, por continuidad la exponencial es medible y la transformada de Fourier de una función en $L^1(\mathbb{R})$ siempre existe\cite{Rudin1991}. 

\item Ya que $\hat{\chi}$ es analítica, el único polo del integrado es $\omega_0$. Ya que el camino está contenido en el conjunto donde el integrando es analítico $\mathbb{R}\times\mathbb{R}^+\setminus\{\omega_0\}$ y no encierra al polo se concluye
\begin{equation}
\int_\mathcal{C}\frac{\hat{\chi}(\omega)}{\omega-\omega_0}\dd{\omega}=0
\end{equation}
por el teorema de Cauchy.

\item Se tiene si $R$ es el radio de $\Gamma$
\begin{equation}
\int_\Gamma\frac{\hat{\chi}(\omega)}{\omega-\omega_0}\dd{\omega}=\int_0^\pi\frac{\hat{\chi}(Re^{i\theta}+\omega_0)}{Re^{i\theta}}iRe^{i\theta}\dd{\theta}=i\int_0^\pi\hat{\chi}(Re^{i\theta}+\omega_0)\dd{\theta}.
\end{equation}
Ahora bien, si $\theta\in(0,\pi)$ entonces $Re^{i\theta}+\omega_0\in\mathbb{R}\times\mathbb{R}^+$ y
\begin{align}\label{ec:cota_1}
\begin{split}
\left|\hat{\chi}(Re^{i\theta}+\omega_0)\right|=&\left|\int_0^\infty\chi(t)e^{i(R\cos(\theta)+\omega_0)t}e^{-R\sin(\theta)t}\dd{t}\right|\\
\leq&\int_0^\infty\left|\chi(t)e^{i(R\cos(\theta)+\omega_0)t}e^{-R\sin(\theta)t}\right|\dd{t}\\
=&\int_0^\infty\left|\chi(t)\right|e^{-R\sin(\theta)t}\dd{t}
\end{split}
\end{align}
La desigualdad es un resultado estándar de la teoría de la medida\cite{Rudin1976}. Además, $\left|\chi\right|\geq\left|\chi(t)\right|e^{-R\sin(\theta)t}$ e integrable. Entonces por el teorema de convergencia dominada\cite{Rudin1976} como $\left|\chi(t)\right|e^{-R\sin(\theta)t}\rightarrow 0$ se tiene que 
\begin{equation}
\left|\hat{\chi}(Re^{i\theta}+\omega_0)\right|\rightarrow 0
\end{equation}
cuando $R\rightarrow\infty$. La positividad de la función medible $(t,\theta)\mapsto\left|\chi(t)\right|e^{-R\sin(\theta)t}$ permite la aplicación del teorema de Fubini\cite{Rudin1987}. Esto hace evidente que la cota \eqref{ec:cota_1} es integrable como función de $\theta$. Aplicando entonces una vez más el teorema de convergencia dominada se obtiene
\begin{equation}
\left|\int_\Gamma\frac{\hat{\chi}(\omega)}{\omega-\omega_0}\dd{\omega}\right|\leq\int_0^\pi\left|\hat{\chi}(Re^{i\theta}+\omega_0)\right|\dd{\theta}\rightarrow 0
\end{equation}
cuando $R\rightarrow\infty$. Por lo tanto
\begin{equation}
\lim_{R\rightarrow\infty}\int_\Gamma\frac{\hat{\chi}(\omega)}{\omega-\omega_0}\dd{\omega}=0.
\end{equation}

Por otra parte, si el radio de $\gamma$ es $\epsilon$
\begin{equation}
\int_\gamma\frac{\hat{\chi}(\omega)}{\omega-\omega_0}\dd{\omega}=i\int_\pi^0\hat{\chi}(\epsilon e^{i\theta}+\omega_0)\dd{\theta}
\end{equation}
Con una cota análoga a \eqref{ec:cota_1} podemos utilizar el teorema de convergencia dominada de manera que la continuidad de $\hat{\chi}$ heredada al ser analítica garantiza que 
\begin{equation}
\int_\gamma\frac{\hat{\chi}(\omega)}{\omega-\omega_0}\dd{\omega}\rightarrow i\int_\pi^0\hat{\chi}(\omega_0)\dd{\theta}=-i\pi\hat{\chi}(\omega_0)
\end{equation}
cuando $\epsilon\rightarrow 0$.

\item Se tiene que
\begin{align}
\begin{split}
P\int_{-\infty}^\infty\frac{\hat{\chi}(\omega)}{\omega-\omega_0}\dd{\omega}=&\\
\lim_{\epsilon\rightarrow 0^+}\qty(\int_{-\infty}^{\omega_0-\epsilon}\frac{\hat{\chi}(\omega)}{\omega-\omega_0}\dd{\omega}+\int_{\omega_0+\epsilon}^\infty\frac{\hat{\chi}(\omega)}{\omega-\omega_0}\dd{\omega})=&\\
\lim_{\epsilon\rightarrow 0^+}\lim_{R\rightarrow\infty}\qty(\int_\mathcal{C}\frac{\hat{\chi}(\omega)}{\omega-\omega_0}\dd{\omega}-\int_\Gamma\frac{\hat{\chi}(\omega)}{\omega-\omega_0}\dd{\omega}-\int_\gamma\frac{\hat{\chi}(\omega)}{\omega-\omega_0}\dd{\omega})=&\\
\lim_{\epsilon\rightarrow 0^+}\lim_{R\rightarrow\infty}\qty(0-\int_\Gamma\frac{\hat{\chi}(\omega)}{\omega-\omega_0}\dd{\omega}+i\pi\hat{\chi}(\omega_0))=\lim_{\epsilon\rightarrow 0^+}i\pi\hat{\chi}(\omega_0)=&\\
i\pi\hat{\chi}(\omega_0)
\end{split}
\end{align}

\item Se deduce que
\begin{equation}
\hat{\chi}(\omega_0)=-\frac{1}{\pi}P\int_{-\infty}^\infty\frac{i\hat{\chi}(\omega)}{\omega-\omega_0}\dd{\omega}.
\end{equation}
Note que la integral respeta suma y multiplicación por escalares, las operaciones necesarias para separar un número complejo en su parte real e imaginaria. Además, tenemos las relaciones
\begin{align}
\begin{split}
\Re(-i(x+iy))=&\Re(y-ix)=y=\Im(x+iy),\\
\Im(-i(x+iy))=&\Im(y-ix)=-x=-\Re(x+iy)
\end{split}
\end{align}
para todo $x,y\in\mathbb{R}$. De estas observaciones se hacen evidentes las relaciones de Kramers-Kronig
\begin{align}
\begin{split}
\Re\hat{\chi}(\omega_0)=&\frac{1}{\pi}P\int_{-\infty}^\infty\frac{\Im\hat{\chi}(\omega)}{\omega-\omega_0}\dd{\omega},\\
\Im\hat{\chi}(\omega_0)=&-\frac{1}{\pi}P\int_{-\infty}^\infty\frac{\Re\hat{\chi}(\omega)}{\omega-\omega_0}\dd{\omega}.
\end{split}
\end{align}

\end{enumerate}

\section{Teorema de fluctuación-disipación cuántico}

\begin{enumerate}

\item Tenemos la ecuación de von Neumann
\begin{align}
\begin{split}
\pdv{\rho}{t} (t)=&\frac{1}{i\hbar}[H(t),\rho(t)]=\frac{1}{i\hbar}[H_0+H_1(t),\rho_eq+\rho_1(t)]\\
=&\frac{1}{i\hbar}\qty([H_0,\rho_{\text{eq}}]+[H_0,\rho_1(t)]+[H_1(t),\rho_{\text{eq}}]+[H_1(t),\rho_1(t)])\\
=&\frac{1}{i\hbar}\qty(\pdv{\rho_{\text{eq}}}{t} (t)+[H_0,\rho_1(t)]+[H_1(t),\rho_{\text{eq}}]+[H_1(t),\rho_1(t)]).
\end{split}
\end{align}
Recordando que $\rho_{\text{eq}}$ es independiente del tiempo e ignorando como demasiado pequeños a los términos de orden 2 en operadores dependientes del tiempo
\begin{equation}
\pdv{\rho_{\text{eq}}}{t} (t)=0=[H_1(t),\rho_1(t)].
\end{equation} 
Se obtiene entonces la ecuación diferencial
\begin{equation}
\pdv{\rho}{t} (t)-[H_0,\rho_1(t)]=\frac{1}{i\hbar}[H_1(t),\rho_{\text{eq}}].
\end{equation}
Esta induce una ecuación diferencial sobre los valores esperados

\end{enumerate}

\bibliography{/Users/ivan/Documents/Bib_Files/library}
%\bibliography{/home/UANDES/im.burbano10/Documents/library}
\bibliographystyle{ieeetr}

\end{document}