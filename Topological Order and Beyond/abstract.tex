\documentclass{article}

\usepackage[utf8]{inputenc}
\usepackage[spanish, english]{babel}

\title{KMS states and Tomita-Takesaki Theory}
\author{Iván Mauricio Burbano Aldana}

\begin{document}

\maketitle

\begin{abstract}

The lack of an axiomatic framework for Quantum Field Theory has brought forward attention to the algebraic formulation of physical theories. The  power of such a description lies on the clearness of the physical and mathematical interpretation of the objects involved. Indeed, the description of thermal equilibrium states finds in the algebraic setting a natural framework through the KMS condition. Here we will discuss this mathematical formulation of equilibrium, its relationship to Gibbs states, and the dynamical invariance it provides. We will then make use of the modular theory of Tomita-Takesaki to show that for every normal faithful state on a von Neumann algebra there is a unique dynamical law such that the state satisfies the KMS condition. We thus conclude that KMS states induce canonical dynamics.

\end{abstract}

\begin{otherlanguage}{spanish}

\begin{abstract}

La falta de un marco axiomático para la Teoría Cuántica de Campos ha aumentado la relevancia de la formulación algebraica de las teorías físicas. Esta descripción es prometedora debido a la transparencia de la interpretación física y matemática de los objetos involucrados. En efecto, la descipción de estados de equilibrio térmico tiene una formulación natural en el marco algebráico denominada la condición KMS. Acá vamos a discutir esta formulación del equilibrio, su relación con los estados de Gibbs y la invarianza dinámica que provee. Luego vamos a utilizar la teoría modular de Tomita-Takesaki para mostrar que para todo estado normal y fiel sobre una álgebra de von Neumann hay una única ley dinámica tal que el estado satisface la condición KMS. Por lo tanto, concluimos que los estados KMS inducen dinámica de manera canónica. 

\end{abstract}

\end{otherlanguage}

\nocite{*}

\bibliography{references}
\bibliographystyle{ieeetr}

\end{document}