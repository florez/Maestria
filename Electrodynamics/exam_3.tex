\documentclass{article}

\usepackage[utf8]{inputenc}
\usepackage{enumerate}
\usepackage{physics}
\usepackage{amssymb}

\title{Electrodynamics\\
Third Exam\\
2018-I}
\author{Iván Mauricio Burbano Aldana}

\begin{document}

\maketitle

\begin{enumerate}[(i)]

\item As we have seen in class the potential $A^\mu$ is given by
\begin{equation}
A^\mu(\vb{r},t)=\frac{\mu_0}{4\pi}\int \dd[3]{\vb{r}'}\frac{j^\mu(\vb{r}',t-\|\vb{r}-\vb{r}'\|/c)}{\|\vb{r}-\vb{r}'\|}
\end{equation}
ignoring all background waves, that is, solutions of $\Box A^\mu=0$. Therefore, its temporal Fourier transform, if we assume exists, is given by
\begin{align}
\begin{split}
A^\mu(\vb{r},\omega)=&\int\dd{t}e^{i\omega t}A^\mu(\vb{r},t)\\
=&\frac{\mu_0}{4\pi}\int\dd{t}e^{i\omega t}\int \dd[3]{\vb{r}'}\frac{j^\mu(\vb{r}',t-\|\vb{r}-\vb{r}'\|/c)}{\|\vb{r}-\vb{r}'\|}\\
=&\frac{\mu_0}{4\pi}\int\dd{t}\int\dd[3]{\vb{r}'}\frac{1}{\|\vb{r}-\vb{r}'\|}e^{i\omega t}j^\mu(\vb{r}',t-\|\vb{r}-\vb{r}'\|/c).
\end{split}
\end{align}
Assuming the proper conditions are met to apply Fubini's theorem we have
\begin{align}
\begin{split}
A^\mu(\vb{r},\omega)=&\frac{\mu_0}{4\pi}\int\dd[3]{\vb{r}'}\int\dd{t}\frac{1}{\|\vb{r}-\vb{r}'\|}e^{i\omega t}j^\mu(\vb{r}',t-\|\vb{r}-\vb{r}'\|/c)\\
=&\frac{\mu_0}{4\pi}\int \dd[3]{\vb{r}'}\frac{1}{\|\vb{r}-\vb{r}'\|}\int\dd{t}e^{i\omega t}j^\mu(\vb{r}',t-\|\vb{r}-\vb{r}'\|/c)
\end{split}
\end{align}
Through the change of variables
\begin{align}
\begin{split}
\mathbb{R}&\rightarrow\mathbb{R}\\
t&\mapsto t+\|\vb{r}-\vb{r}'\|/c
\end{split}
\end{align}
which keeps the domain $\mathbb{R}$ invariant we obtain
\begin{align}
\begin{split}
A^\mu(\vb{r},\omega)=&\frac{\mu_0}{4\pi}\int \dd[3]{\vb{r}'}\frac{1}{\|\vb{r}-\vb{r}'\|}\int\dd{t}e^{i\omega(t+\|\vb{r}-\vb{r}'\|/c)}j^\mu(\vb{r}',t)\\
=&\frac{\mu_0}{4\pi}\int \dd[3]{\vb{r}'}\frac{1}{\|\vb{r}-\vb{r}'\|}\int\dd{t}e^{i\omega\|\vb{r}-\vb{r}'\|/c}e^{i\omega t}j^\mu(\vb{r}',t)\\
=&\frac{\mu_0}{4\pi}\int \dd[3]{\vb{r}'}\frac{e^{iK\|\vb{r}-\vb{r}'\|}}{\|\vb{r}-\vb{r}'\|}\int\dd{t}e^{i\omega t}j^\mu(\vb{r}',t).
\end{split}
\end{align}
We now recognize the Fourier transform
\begin{equation}\label{ec:fourier_current}
j^\mu(\vb{r},\omega)=\int\dd{t}e^{i\omega t}j^\mu(\vb{r},t)
\end{equation}
concluding
\begin{equation}
A^\mu(\vb{r},\omega)=\frac{\mu_0}{4\pi}\int \dd[3]{\vb{r}'}\frac{e^{iK\|\vb{r}-\vb{r}'\|}}{\|\vb{r}-\vb{r}'\|}j^\mu(\vb{r}',\omega).
\end{equation}
Separating this equation by remembering that $A^\mu=(\phi/c,\vb{A})$, $j^\mu=(c\rho,\vb{J})$, and $c^2\mu_0=\frac{\mu_0}{\mu_0\epsilon_0}=\frac{1}{\epsilon_0}$ we obtain
\begin{align}\label{ec:fourier_vector_potential}
\begin{split}
\phi(\vb{r},\omega)=&\frac{1}{4\pi\epsilon_0}\int \dd[3]{\vb{r}'}\frac{e^{iK\|\vb{r}-\vb{r}'\|}}{\|\vb{r}-\vb{r}'\|}\rho(\vb{r}',\omega),\\
\vb{A}(\vb{r},\omega)=&\frac{\mu_0}{4\pi}\int \dd[3]{\vb{r}'}\frac{e^{iK\|\vb{r}-\vb{r}'\|}}{\|\vb{r}-\vb{r}'\|}\vb{J}(\vb{r}',\omega).
\end{split}
\end{align}

\item Recall that the law of charge current conservation is given by
\begin{equation}
\partial_\mu j^\mu(\vb{r},t)=0.
\end{equation}
Assuming that the Fourier transformation \eqref{ec:fourier_current} is invertible, we must have
\begin{equation}
j^\mu(\vb{r},t)=\frac{1}{2\pi}\int\dd{\omega}e^{-i\omega t}j^\mu(\vb{r},\omega)
\end{equation}
and
\begin{equation}
0=\frac{1}{2\pi}\partial_\mu\int\dd{\omega}e^{-i\omega t}j^\mu(\vb{r},\omega).
\end{equation}
Assuming that the right conditions are met to interchange differentiation with integration we have
\begin{align}
\begin{split}
0=&\int\dd{\omega}\partial_\mu\qty(e^{-i\omega t}j^\mu(\vb{r},\omega))\\
=&\int\dd{\omega}\qty(\frac{1}{c}\pdv{e^{-i\omega t}}{t}c\rho(\vb{r},\omega)+e^{-i\omega t}\div{\vb{J}}(\vb{r},\omega))\\
=&\int\dd{\omega}\qty(-i\omega e^{-i\omega t}\rho(\vb{r},\omega)+e^{-i\omega t}\div{\vb{J}}(\vb{r},\omega))\\
=&\int\dd{\omega}e^{-i\omega t}\qty(-i\omega\rho(\vb{r},\omega)+\div{\vb{J}}(\vb{r},\omega)).
\end{split}
\end{align}
Under the right conditions the uniqueness theorem is valid and a function is null if and only if is Fourier transform is. We thus conclude that the law of charge current conservation can be expressed as
\begin{equation}\label{ec:fourier_charge_current}
\div{\vb{J}}(\vb{r},\omega)=i\omega\rho(\vb{r},\omega).
\end{equation}

\item We recall the definition of the electric dipole moment and magnetic dipole moment
\begin{align}
\begin{split}
\vb{p}=&\int\dd[3]{\vb{r}}\vb{r}\rho(\vb{r}),\\
\vb{m}=&\frac{1}{2}\int\dd[3]{\vb{r}}\vb{r}\cross\vb{J}(\vb{r}).
\end{split}
\end{align}
By using the product rule of the divergence
\begin{align}
\begin{split}
\int\dd[3]{\vb{r}}\vb{J}(\vb{r},\omega)=&\sum_{i=1}^3\vu{e}_i\int\dd[3]{\vb{r}}\vb{J}(\vb{r},\omega)\vdot\vu{e}_{i}\\
=&\sum_{i=1}^3\vu{e}_i\int\dd[3]{\vb{r}}\vb{J}(\vb{r},\omega)\vdot\grad{x^i}\\
=&\sum_{i=1}^3\vu{e}_i\int\dd[3]{\vb{r}}(\div(x^i\vb{J})(\vb{r},\omega)-x^i\div{\vb{J}}(\vb{r},\omega))\\
=&\sum_{i=1}^3\vu{e}_i\qty(\int\dd[3]{\vb{r}}\div(x^i\vb{J})(\vb{r},\omega)-\int\dd[3]{\vb{r}}x^i\div{\vb{J}}(\vb{r},\omega)).
\end{split}
\end{align}
The first integral vanishes since it is a total differential on a region without a boundary, namely $\mathbb{R}^3$. Therefore, by using \eqref{ec:fourier_charge_current} we obtain
\begin{align}\label{ec:integral_current}
\begin{split}
\int\dd[3]{\vb{r}}\vb{J}(\vb{r},\omega)=&-\sum_{i=1}^3\vu{e}_i\int\dd[3]{\vb{r}}x^i\div{\vb{J}}(\vb{r},\omega)\\
=&-\int\dd[3]{\vb{r}}\sum_{i=1}^3\vu{e}_ix^ii\omega\rho(\vb{r},\omega)=-\int\dd[3]{\vb{r}}\vb{r}i\omega\rho(\vb{r},\omega)\\
=&-i\omega\vb{p}.
\end{split}
\end{align}

In the far zone we have as hinted in the problem sheet
\begin{equation}
\frac{
1}{\|\vb{r}-\vb{r}'\|}\approx\frac{
1}{\|\vb{r}\|-\frac{\vb{r}\vdot\vb{r}'}{\|\vb{r}\|}}=\frac{
1}{r-\frac{\vb{r}\vdot\vb{r}'}{r}}=\frac{1}{r}\frac{1}{1-\frac{\vb{r}\vdot\vb{r}'}{r^2}}
\end{equation}
where $r:=\|\vb{r}\|$ to lighten notation. Given that
\begin{equation}\label{ec:estimate_1}
\left|\frac{\vb{r}\vdot\vb{r}'}{r^2}\right|\leq\frac{rr'}{r^2}=\frac{r'}{r}
\end{equation}
if we assume that the fields are local, that is, that $r'\ll r$ in the region where fields are relevant we may say that under the integral sign $|r'/r|\ll 1$. Therefore, we may employ the geometric series
\begin{equation}
\frac{1}{1-x}=\sum_{n=0}^\infty x^n
\end{equation}
to approximate
\begin{equation}
\frac{
1}{\|\vb{r}-\vb{r}'\|}\approx\frac{1}{r}\qty(1+\frac{\vb{r}\vdot\vb{r}'}{r^2}).
\end{equation}
Similarly, we may stablish the estimate
\begin{equation}
e^{ik\|\vb{r}-\vb{r}'\|}\approx e^{ik\qty(r-\frac{\vb{r}\vdot\vb{r}'}{r})}=e^{ikr}e^{-ik\frac{\vb{r}\vdot\vb{r}'}{r}}=e^{ikr}\sum_{n=0}^\infty\frac{(-ik)^n}{n!}\qty(\frac{\vb{r}\vdot\vb{r}'}{r})^n.
\end{equation}
Thus,
\begin{align}
\begin{split}
\frac{
e^{ik\|\vb{r}-\vb{r}'\|}}{\|\vb{r}-\vb{r}'\|}\approx&\frac{1}{r}\qty(1+\frac{\vb{r}\vdot\vb{r}'}{r^2})e^{ikr}\sum_{n=0}^\infty\frac{(-ik)^n}{n!}\qty(\frac{\vb{r}\vdot\vb{r}'}{r})^n\\
=&\qty(1+\frac{\vb{r}\vdot\vb{r}'}{r^2})e^{ikr}\sum_{n=0}^\infty\frac{(-ik)^n}{n!}\frac{1}{r}\qty(\frac{\vb{r}\vdot\vb{r}'}{r})^n.
\end{split}
\end{align}
Due to estimate \eqref{ec:estimate_1} we can truncate the series to obtain
\begin{equation}
\frac{
e^{ik\|\vb{r}-\vb{r}'\|}}{\|\vb{r}-\vb{r}'\|}\approx\qty(1+\frac{\vb{r}\vdot\vb{r}'}{r^2})e^{ikr}\qty(\frac{1}{r}-ik\frac{\vb{r}\vdot\vb{r}'}{r^2}).
\end{equation}
Thus far, all of our estimates have been consistent up to $\mathcal{O}(\frac{\vb{r}\vdot\vb{r}'}{r^2})$. Therefore, we may estimate further
\begin{equation}
\frac{e^{ik\|\vb{r}-\vb{r}'\|}}{\|\vb{r}-\vb{r}'\|}\approx e^{ikr}\qty(\frac{1}{r}-ik\frac{\vb{r}\vdot\vb{r}'}{r^2})=\frac{e^{ikr}}{r}\qty(1-ik\frac{\vb{r}\vdot\vb{r}'}{r}).
\end{equation}
Pluggin this estimate in \eqref{ec:fourier_vector_potential} we obtain
\begin{equation}
\vb{A}(\vb{r},\omega)=\frac{\mu_0e^{ikr}}{4\pi r}\int \dd[3]{\vb{r}'}\qty(1-ik\frac{\vb{r}\vdot\vb{r}'}{r})\vb{J}(\vb{r}',\omega).
\end{equation}
Remembering \eqref{ec:integral_current} and the hint in the exam sheet we obtain
\begin{align}
\begin{split}
\vb{A}(\vb{r},\omega)=&\frac{\mu_0e^{ikr}}{4\pi r}\qty(-i\omega\vb{p}-\frac{ik}{r}\int\dd[3]{\vb{r}'}(\vb{r}\vdot\vb{r}')\vb{J}(\vb{r}',\omega))\\
=&-\frac{\mu_0e^{ikr}}{4\pi r}\qty(i\omega\vb{p}+\frac{ik}{2r}\int\dd[3]{\vb{r}'}(\vb{r}'\cross\vb{J}(\vb{r}',\omega))\cross\vb{r})\\
=&-\frac{\mu_0e^{ikr}}{4\pi r}\qty(i\omega\vb{p}+\frac{ik}{r}\vb{m}\cross\vb{r}).
\end{split}
\end{align}

\item We can express the Fourier transform of the electric and magnetic field by
\begin{equation}
\vb{B}(\vb{r},\omega)=\int\dd{t}e^{i\omega t}\vb{B}(\vb{r},t)=\int\dd{t}e^{i\omega t}\curl{\vb{A}}(\vb{r},t).
\end{equation}
Given that $e^{i\omega t}$ has no spatial dependence and assuming that the right conditions are met to exchange differentiation and integration we obtain
\begin{equation}
\vb{B}(\vb{r},\omega)=\curl{\int\dd{t}e^{i\omega t}\vb{A}(\vb{r},t)}=\curl{\vb{A}}(\vb{r},\omega).
\end{equation}
Making use of the product rule we have by inserting 

\end{enumerate}

\end{document}